\documentclass[compress]{beamer}
\mode<presentation>

% Include additional LaTeX packages
\usepackage{subfigure}
\usepackage{multicol}
\usepackage{amsmath}
\usepackage{epsfig}
\usepackage{graphicx}
\usepackage[all,knot]{xy}
\xyoption{arc}
\usepackage{url}
\usepackage{multimedia}
\usepackage{hyperref}
\usepackage{setspace}

% Set up Beamer theme
\usetheme{Dresden}
\usecolortheme{lily}
\usefonttheme{structuresmallcapsserif}
\usepackage{beamerinnerthemecircles}
\usepackage{beamerouterthememiniframes}
\useoutertheme[subsection=false]{smoothbars}

% Removes the Beamer navigation symbols
\setbeamertemplate{navigation symbols}{}

% Sets the color for the \alert{} text
\setbeamercolor{alerted text}{fg=blue}

% Presentation information
\title[Linux Orientation Workshop -- \insertframenumber/\inserttotalframenumber]{Linux Orientation Workshop \vspace{.20cm} \hrule}
\subtitle{For Second Year Computer Engineering Students}
\author[\copyright 2012, Reverse Bit Coders]{Akshay Mankar, Abdulkarim Memon, Imran Ahmed, Miheer Vaidya, Prathamesh Sonpatki}
\institute{Reverse Bit Coders}
\date{\tiny 27th Jun 2012}

\begin{document}

% Create title page
\frame{\maketitle}

% Create table of contents
\frame{\frametitle{Outline}\tableofcontents}

\section{Linux Basics}
\subsection{What is Linux}
\frame{\frametitle{What is Linux}
  \begin{itemize}
    \item Linux is a \alert {Unix-like} computer \alert {operating system} assembled under the model of \alert {free and open source software} development and distribution.\footnotemark
    \item Available in various \alert{Distros} bundled with various other free software
    \begin{itemize}
    	\item Ubuntu
    	\item Fedora
    	\item Linux Mint
    	\item ...
    \end{itemize}
    \item Started by \alert{Linus Torvalds} in 1991
  \end{itemize}
  \footnotetext {http://en.wikipedia.org/wiki/Linux}
}

\subsection{The Filesystem}
\frame{\frametitle{The Filesystem}
  \begin{itemize}
  	\item Follows \alert {Filesystem Hierarchy Standard (FHS)}
  	\item \alert {/} : Root directory of entire filesystem
  	\item \alert {/bin} : Essential command binaries
  	\item \alert {/dev} : Essential Devices
  	\item \alert {/etc} : System wide configurations
  	\item \alert {/home} : User's home directories
  	\item \alert {/lib} : Libraries essential for binaries
  	\item \alert {/media} : Mount points for removable media
  	\item \alert {/mnt} : Temporary mounted filesystems
  	\item \alert {/opt} : Optional Applications
  	\item \alert {/root} : Home directory for root user
  	\item \alert {/sbin} : Essential System Binaries
  	\item \alert {/tmp} : Temporary files
  	\item \alert {/usr} : Secondary hierarchy for read-only user data
  	\item \alert {/var} : Variable Files
  \end{itemize}
}
\frame{\frametitle{Screenshot}
}

\subsection{three}
\frame{\frametitle{three}
	
}
\section{Second}
\subsection{Sub-First}
\frame{\frametitle{Second}
  New section
}

\subsection{Sub-Second}
\frame{\frametitle{Sub-Second}
  Next sub-section
}

% Ending slide is title page again
\section*{}
\frame{\titlepage}

\end{document}
